\documentclass[10pt,a4paper,ragged2e,withhyper]{altacv}

%% AltaCV uses the fontawesome5 and academicons fonts
%% and packages.
%% See http://texdoc.net/pkg/fontawesome5 and http://texdoc.net/pkg/academicons for full list of symbols. You MUST compile with XeLaTeX or LuaLaTeX if you want to use academicons.

% Change the page layout if you need to
\geometry{left=1.25cm,right=1.25cm,top=1.5cm,bottom=1.5cm,columnsep=1.2cm}

\sloppy
\hyphenpenalty=10000

% The paracol package lets you typeset columns of text in parallel
\usepackage{paracol}

\usepackage{setspace}

% Change the font if you want to, depending on whether
% you're using pdflatex or xelatex/lualatex
\ifxetexorluatex{}
  % If using xelatex or lualatex:
  \setmainfont{Roboto Slab}
  \setsansfont{Lato}
  \renewcommand{\familydefault}{\sfdefault}
\else
  % If using pdflatex:
  \usepackage[rm]{roboto}
  \usepackage[defaultsans]{lato}
  % \usepackage{sourcesanspro}
  \renewcommand{\familydefault}{\sfdefault}
\fi

% Change the colours if you want to
\definecolor{SlateGrey}{HTML}{2E2E2E}
\definecolor{LightGrey}{HTML}{555555}
\definecolor{DarkPastelBlue}{HTML}{083f5d}
\definecolor{PastelBlue}{HTML}{0b4f70}
\colorlet{name}{black}
\colorlet{tagline}{PastelBlue}
\colorlet{heading}{DarkPastelBlue}
\colorlet{headingrule}{DarkPastelBlue}
\colorlet{subheading}{PastelBlue}
\colorlet{accent}{PastelBlue}
\colorlet{emphasis}{SlateGrey}
\colorlet{body}{LightGrey}

% Change some fonts, if necessary
\renewcommand{\namefont}{\huge\rmfamily\bfseries}
\renewcommand{\personalinfofont}{\footnotesize}
\renewcommand{\cvsectionfont}{\Large\rmfamily\bfseries}
\renewcommand{\cvsubsectionfont}{\large\bfseries}


% Change the bullets for itemize and rating marker
% for \cvskill if you want to
\renewcommand{\itemmarker}{{\small\textbullet}}
\renewcommand{\ratingmarker}{\faCircle}

%% sample.bib contains your publications
\addbibresource{publications.bib}

\begin{document}
\name{Savindi Wijenayaka}
\tagline{Machine Learning Engineer \& Researcher}
%% You can add multiple photos on the left or right
% \photoR{3cm}{profile-photo}

\personalinfo{%
  \expandafter\ifstrequal\expandafter{\jobname}{savindi-cv}
  {
  \printinfo{\faMapMarker*}{Auckland, New Zealand}
  \printinfo{\faGlobe}{https://savindi.carrd.co}[savindi.carrd.co]\\
  \printinfo{\faLinkedin}{https://linkedin.com/in/savindi}[https://linkedin.com/in/savindi]
  \printinfo{\faGithub}{https://github.com/savindi-wijenayaka}[https://github.com/savindi-wijenayaka]
  }
  {
  \printinfo{\faPhone}{+64 22 453 8372}
  \printinfo{\faAt}{sabe848@aucklanduni.ac.nz}[mailto:sabe848@aucklanduni.ac.nz]
  \printinfo{\faMapMarker*}{Auckland, New Zealand}\\
  \printinfo{\faGlobe}{savindi.carrd.co}[https://savindi.carrd.co]
  \printinfo{\faLinkedin}{linkedin.com/in/savindi}[https://linkedin.com/in/savindi]
  \printinfo{\faMedium}{savindi-wijenayaka.medium.com}[https://savindi-wijenayaka.medium.com]
%   \printinfo{\faGithub}{https://github.com/savindi-wijenayaka}[https://github.com/savindi-wijenayaka]
  }

  %% You can add your own arbtrary detail with
  %% \printinfo{symbol}{detail}[optional hyperlink prefix]
}

\makecvheader{}
%% Depending on your tastes, you may want to make fonts of itemize environments slightly smaller
% \AtBeginEnvironment{itemize}{\small}

\medskip

\cvsection{Summary}

A Machine Learning Engineer and a Researcher with 2+ years of experience in applied machine learning research and production-grade cloud-native application development. Currently working towards a Ph.D. in Bioengineering to contribute to the advancement of healthcare with the aid of Artificial Intelligence (AI). Seeing someone's face light up with a smile due to a product I helped create brings me ultimate satisfaction as an AI and tech enthusiast.

\medskip

\cvsection{Education}

\cvevent{Ph.D. in Bioengineering}{University of Auckland}{Dec 2021 - Present}{Auckland, New Zealand}   % chktex 8
\begin{itemize}
    \item Analyse microstructures of upper gastrointestinal (GI) sphincters and develop computational models using novel imaging techniques and deep learning to improve understanding of GI disorders and benefit in silico experiments. %develop new treatments
    % \item Developed and refined innovative machine learning pipeline to aid segmentation of anatomical structures in the stomach, achieving initial dice scores of 94.5\% and 81.9\% on test and new data, respectively.
    % \item Quantified the orientations of muscle fibres within sphincters, providing a unique perspective to study functional disturbances associated with upper gastrointestinal tract disorders and resulting in the identification of key structural features that were used to develop a unified model.
    % \item Collaborated with a multidisciplinary team of bioinformaticians and data scientists to develop an open-source platform that facilitates data sharing across institutions and promotes transparency in biomedical research. The platform has already been adopted by multiple labs at NIH, demonstrating its efficacy as an innovative tool for advancing scientific discoveries.
\end{itemize}

\divider{}

\cvevent{B.Sc. (Hons.) in Software Engineering}{University of Kelaniya}{Feb 2016 - March 2020}{Kelaniya, Sri Lanka}   % chktex 8
\begin{itemize}
    \item Specialised in Data Science and Net-centric application development
    \item Attained a GPA of 3.96 out of 4.00, obtaining a First Class.
\end{itemize}

\medskip

\cvsection{Experience}

\cvevent{Machine Learning Engineer}{\href{http://www.wso2.com}{WSO2}}{Sept 2020 - Nov 2021}{Colombo, Sri Lanka}   % chktex 8
\begin{itemize}
    \item WSO2 is one of the world's leading open-source integration vendors. Choreo is its latest product providing an AI-enhanced integrated platform as a service. 
    \item Researched, engineered and deployed the initial phase of Choreo's AI-assisted testing feature, using Python, Keras, Flask, Kubernetes and Azure DevOps pipelines.
    \item Architected, developed and deployed Choreo’s AI-based anomaly detector with two other engineers, using Azure solutions, Ballerina, and Python, while adhering to security best practices, scaling requirements, and optimised resource usage.
    \item Analysed Ballerina Language Server performance and identified the cause of a memory leak using JMeter and Eclipse Memory Analyser (MAT), which helped in the optimisation of resources in Choreo.
    \item Contributed to automating the performance testing of Choreo by creating a library and a pipeline for system metrics collection using Python, Kusto, Seaborn, and Azure DevOps pipelines.
\end{itemize}

\divider{}

\cvevent{Software Research Engineer (Intern)}{\href{https://www.pearson.com/}{Pearson}}{Sept 2018 - Sept 2019}{Colombo, Sri Lanka}   % chktex 8
\begin{itemize}
    \item Pearson is a leading Education provider, offering curriculum materials, multimedia learning tools, and testing programs to help educate people worldwide.
    \item Collaborated with two other engineers to create the minimum viable product of AI-based Public Speaking Evaluator Service (APSES) while contributing to emotion detection and speech analysis features, using Python, Keras, OpenCV, Kaldi and Flask.
    \item Investigated on Question and Answering and built the minimal viable product of a Chatbot, which answers students' questions based on Pearson books and other documentation, using a modified version of the Bi-Directional Attention Flow (BiDAF) model, Python and Django.
    \item Researched and engineered the minimal viable product which automatically classifies flashcards created by the system or users under available topics, using the Universal Language Model Fine-Tuning (ULMFiT) model, Python and Django.
\end{itemize}

\medskip

% \pagebreak
\cvsection{Certifications}

\begin{itemize}
    \item \cvevent{AI for Medical Diagnosis}{DeepLearning.AI}{May 2021}{}
    \item \cvevent{Deep Learning Specialisation}{DeepLearning.AI}{December 2020}{}
    \item \cvevent{TensorFlow Developer Specialisation}{DeepLearning.AI}{July 2020}{}
\end{itemize}

\medskip

\cvsection{Skills}

\begin{itemize}
    \item \textbf{Knowledge Areas } --- Deep learning (Vision \& NLP)
    \item \textbf{Programming languages} --- Python, Ballerina, Java SE, Bash
    \item \textbf{Frameworks and tools} --- Flask, TensorFlow, Numpy, Pandas, Keras, Springboot, Git, Agile
    \item \textbf{DevOps} --- Linux, Azure, Kubernetes, Docker, Kustomize, JMeter
\end{itemize}

\medskip

\cvsection{Achievements}
\begin{itemize}
    \item Awarded as the 1st place team in SPARC FAIR codeathon 2022 organized jointly by SPARC Data and Resource Center and National Institutes of Health (NIH)
    \item Placements in Dean’s List in all four academic years of B.Sc.
    \item Awarded black belt in Karate.
    \item Member (2016-2018) and women's team captain (2018) in the University of Kelaniya Karate team, winning national and inter-university medals with merits. 
    \item Lead the Marketing and Communication function of AIESEC in University of Kelaniya as the vice president (2018) to increase the growth rate.
    \item Managed the financial and budgeting aspects as the organising committee finance lead in RealHack 2019 inter-university hackathon.
    \item Awarded as the 4th Place team in the DataStorm 2020 datathon organised by Octave (JKH Centre of Excellence for Big Data Analytics) and University of Moratuwa
    \item Became the 1st Runner Up team in National Youth Software Competition 2017 organised by UNDP-Sri Lanka.
    \item Achieved the Gold medal for Chess in Four Nations Championship 2019 organized by Pearson.
    \item Represented secondary school at the national level for Karate, Carrom, Kabbadi and as an Army Cadet.
    \item Held various leadership positions in the secondary school as a junior prefect, senior prefect and Karate captain.
\end{itemize}

\medskip

\cvsection{Volunteer and Webinar hosting}
\begin{itemize}
    \item Member of the teaching team for Code In Place 2021, an online Python course offered by Stanford University during the COVID-19 pandemic. 
    \item Guest speaker of IEEE Hobnobbers 2021, sharing knowledge on the topic ``A dive into deep learning. ``
    \item Guest speaker of Pie \& AI Sri Lankan session, organised by DeepLearning.AI, on the topic ``An Introduction to AI and Machine Learning - Sinhala. ``
\end{itemize}

\medskip

\cvsection{Languages}

\cvskill{English}{4}
\cvskill{Sinhala}{5}

\vfill
\begin{flushright}
\flushright
\footnotesize{\emph{Last updated: ~\today }}
\end{flushright}

\end{document}
