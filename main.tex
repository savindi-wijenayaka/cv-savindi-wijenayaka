\documentclass[12pt,a4paper,withhyper]{altacv}
% \justifying

%% AltaCV uses the fontawesome5 and academicons fonts
%% and packages.
%% See http://texdoc.net/pkg/fontawesome5 and http://texdoc.net/pkg/academicons for full list of symbols. You MUST compile with XeLaTeX or LuaLaTeX if you want to use academicons.
% 2.54 cm
% Change the page layout if you need to
\geometry{left=2cm,right=1.75cm,top=1.75cm,bottom=1.75cm}

\sloppy
% \hyphenpenalty=100
% \usepackage[none]{hyphenat}

% The paracol package lets you typeset columns of text in parallel
% \usepackage{paracol}
% \usepackage{setspace}

% Change the font if you want to, depending on whether
% you're using pdflatex or xelatex/lualatex
\ifxetexorluatex{}
  % If using xelatex or lualatex:
  \setmainfont{Roboto Slab}
  \setsansfont{Lato}
  \renewcommand{\familydefault}{\sfdefault}
\else
  % If using pdflatex:
  \usepackage[rm]{roboto}
  \usepackage[defaultsans]{lato}
  % \usepackage{sourcesanspro}
  \renewcommand{\familydefault}{\sfdefault}
\fi

% Change the colours if you want to
\definecolor{SlateGrey}{HTML}{2E2E2E}
\definecolor{LightGrey}{HTML}{555555}
\definecolor{DarkPastelBlue}{HTML}{083f5d}
\definecolor{PastelBlue}{HTML}{0b4f70}
\colorlet{name}{black}
\colorlet{tagline}{PastelBlue}
\colorlet{heading}{DarkPastelBlue}
\colorlet{headingrule}{DarkPastelBlue}
\colorlet{subheading}{PastelBlue}
\colorlet{accent}{PastelBlue}
\colorlet{emphasis}{SlateGrey}
\colorlet{body}{LightGrey}

% Change some fonts, if necessary
\renewcommand{\namefont}{\huge\rmfamily\bfseries}
\renewcommand{\personalinfofont}{\footnotesize}
\renewcommand{\cvsectionfont}{\Large\rmfamily\bfseries}
\renewcommand{\cvsubsectionfont}{\large\bfseries}


% Change the bullets for itemize and rating marker
% for \cvskill if you want to
\renewcommand{\itemmarker}{{\small\textbullet}}
\renewcommand{\ratingmarker}{\faCircle}

% Contains your publications
% \addbibresource{publications.bib}

\begin{document}
\name{Savindi Wijenayaka}
\tagline{Machine Learning Engineer \& Researcher}
%% You can add multiple photos on the left or right
% \photoR{3cm}{profile-photo}

\personalinfo{%
  {
  \printinfo{\faPhone}{+64 22 453 8372}
  \printinfo{\faAt}{sabe848@aucklanduni.ac.nz}[mailto:sabe848@aucklanduni.ac.nz]
  \printinfo{\faMapMarker*}{Auckland, New Zealand}\\
  \printinfo{\faGlobe}{savindi.com}[https://savindi.com]
  \printinfo{\faLinkedin}{linkedin.com/in/savindi}[https://linkedin.com/in/savindi]
  \printinfo{\faMedium}{savindi-wijenayaka.medium.com}[https://savindi-wijenayaka.medium.com]
  }
}

\makecvheader{}

\medskip

\cvsection{Summary}

Machine Learning Engineer and Researcher with over two years of experience in applied deep learning and developing scalable, production-grade cloud-native applications. PhD in Bioengineering, currently under examination, with over three years of interdisciplinary research combining medical imaging, computational quantification, and deep learning-based analysis. Driven by a passion for building AI solutions with real-world impact and committed to bridging the gap between research and application.

\medskip

\cvsection{Education}

\cvevent{Ph.D. in Bioengineering \small{(under examination)}}{University of Auckland}{Dec 2021 - May 2025}{Auckland, New Zealand}   
\begin{itemize}
    \item Analysed gastric microstructures using an interdisciplinary approach integrating micro-CT imaging, mathematics, computational anatomy, and deep learning, advancing the understanding of healthy stomachs.
    \item Engineered a semantic segmentation model to distinguish gastric tissue layers, integrating multiscale channel and spatial attention concepts, implementing numerous ablation studies, and saving over 40 hours per dataset.
    \item Developed a robust three-dimensional gastric tissue quantification framework using advanced mathematical techniques in Python; delivered precise measurements from 20+ tissue samples, establishing the first benchmark for future research.
    \item Developed a comprehensive computational model compiling crucial geometric information alongside quantification details sourced from 8 distinct experiments, enabling future in-silico experiments.
    \item Conducted biological experiments to collect and prepare rodent stomachs for micro CT imaging, resulting in a streamlined process that enhanced sample quality and consistency across 15 experimental trials.
\end{itemize}

\divider{}

\cvevent{B.Sc. (Hons.) in Software Engineering}{University of Kelaniya}{Feb 2016 - Mar 2020}{Kelaniya, Sri Lanka}   % chktex 8
\begin{itemize}
    \item Specialised in Data Science and Net-centric application development
    \item Attained a GPA of 3.96 out of 4.00, obtaining a First Class.
\end{itemize}

\medskip

\cvsection{Experience}

\cvevent{Machine Learning Engineer}{\href{http://www.wso2.com}{WSO2}}{Sept 2020 - Nov 2021}{Colombo, Sri Lanka}   % chktex 8
\begin{itemize}
    \item WSO2 is one of the world's leading open-source integration vendors. Choreo is its latest product, providing an AI-enhanced integrated platform as a service. 
    \item Researched, engineered and deployed the initial phase of Choreo's AI-assisted testing feature, using Python, Keras, Flask, Kubernetes and Azure DevOps pipelines.
    \item Architected, developed and deployed Choreo’s AI-based anomaly detector with two other engineers, using Azure solutions, Ballerina, and Python, while adhering to security best practices, scaling requirements, and optimised resource usage.
    \item Analysed Ballerina Language Server performance and identified the cause of a memory leak using JMeter and Eclipse Memory Analyser (MAT), which helped in the optimisation of resources in Choreo.
    \item Contributed to automating the performance testing of Choreo by creating a library and a pipeline for system metrics collection using Python, Kusto Query Language (KQL), Seaborn, and Azure DevOps pipelines.
\end{itemize}

% \divider{}
\clearpage
\cvevent{Software Research Engineer (Intern)}{\href{https://www.pearson.com/}{Pearson}}{Sept 2018 - Sept 2019}{Colombo, Sri Lanka}   % chktex 8
\begin{itemize}
    \item Pearson is a leading Education provider, offering curriculum materials, multimedia learning tools, and testing programs to help educate people worldwide.
    \item Collaborated with two other engineers to create the minimum viable product of AI-based Public Speaking Evaluator Service (APSES) while contributing to emotion detection and speech analysis features, using Python, Keras, OpenCV, Kaldi and Flask.
    \item Investigated on Question and Answering and built the minimal viable product of a Chatbot, which answers students' questions based on Pearson books and other documentation, using a modified version of the Bi-Directional Attention Flow (BiDAF) model, Python and Django.
    \item Researched and engineered the minimal viable product which automatically classifies flashcards created by the system or users under available topics, using the Universal Language Model Fine-Tuning (ULMFiT) model, Python and Django.
\end{itemize}

\medskip

% % \pagebreak
% \cvsection{Certifications}

% \begin{itemize}
%     \item \cvevent{AI for Medical Diagnosis}{DeepLearning.AI}{May 2021}{}
%     \item \cvevent{Deep Learning Specialisation}{DeepLearning.AI}{Dec 2020}{}
%     \item \cvevent{TensorFlow Developer Specialisation}{DeepLearning.AI}{July 2020}{}
% \end{itemize}

\medskip

\cvsection{Skills}

\begin{itemize}
    \item \textbf{Knowledge Areas } --- Deep learning (Vision \& NLP)
    \item \textbf{Programming languages} --- Python, Ballerina, Java SE, Bash
    \item \textbf{Frameworks and tools} --- Flask, Pytorch, Keras, Numpy, Pandas, Springboot, Git, Agile
    \item \textbf{DevOps} --- Linux, Azure, Kubernetes, Docker, Kustomize, AWS, JMeter
\end{itemize}

\medskip

\cvsection{Achievements}
\begin{itemize}
    % Technical / Academic
    \item \textbf{1st Place (2022)} and \textbf{2nd Place (2024)} – SPARC FAIR Codeathon, organised by SPARC Data and Resource Centre and the NIH.
    \item \textbf{4th Place} – DataStorm 2020 Datathon, organised by Octave (JKH) and University of Moratuwa.
    \item \textbf{1st Runner-Up} – National Youth Software Competition 2017, organised by UNDP Sri Lanka.
    \item \textbf{Dean’s List Honouree} – Recognised in all four academic years of the B.Sc. programme.

    % Leadership & Management
    \item \textbf{Vice President} – Marketing \& Communications at AIESEC in the University of Kelaniya in 2018, contributing to local chapter growth.

    % Sports & Extracurricular
    \item \textbf{National University Colours} recipient, \textbf{Black Belter} and \textbf{Women's Captain} in University Karate Team; won multiple national and inter-university medals (2016–2018).

    \item \textbf{Gold Medalist} – Chess, Four Nations Championship 2019, organised by Pearson.
    \item Represented school at the national level in Karate, Carrom, Kabaddi, and Army Cadet competitions, while also serving as Junior Prefect, Senior Prefect, and Karate Captain.

\end{itemize}



\medskip

\cvsection{Volunteer and Webinar hosting}
\begin{itemize}
    \item Member of the teaching team for Code In Place 2021, an online Python course offered by Stanford University during the COVID-19 pandemic. 
    \item Guest speaker of IEEE Hobnobbers 2021, sharing knowledge on the topic ``A dive into deep learning. ``
    \item Guest speaker of Pie \& AI Sri Lankan session, organised by DeepLearning.AI, on the topic ``An Introduction to AI and Machine Learning - Sinhala. ``
\end{itemize}

\medskip

\cvsection{Languages}

\cvskill{English}{4}
\cvskill{Sinhala}{5}

\medskip
\cvsection{References}
\textit{Available upon request.}

% \vfill
% \begin{flushright}
% \flushright
% \footnotesize{\emph{Last updated: ~\today }}
% \end{flushright}

\end{document}
