\documentclass[12pt,a4paper,withhyper]{altacv}
% \justifying

%% AltaCV uses the fontawesome5 and academicons fonts
%% and packages.
%% See http://texdoc.net/pkg/fontawesome5 and http://texdoc.net/pkg/academicons for full list of symbols. You MUST compile with XeLaTeX or LuaLaTeX if you want to use academicons.
% 2.54 cm
% Change the page layout if you need to
\geometry{left=2cm,right=2cm,top=2cm,bottom=2cm}

\sloppy
% \hyphenpenalty=100
% \usepackage[none]{hyphenat}

% The paracol package lets you typeset columns of text in parallel
% \usepackage{paracol}
% \usepackage{setspace}

% Change the font if you want to, depending on whether
% you're using pdflatex or xelatex/lualatex
\ifxetexorluatex{}
  % If using xelatex or lualatex:
  \setmainfont{Roboto Slab}
  \setsansfont{Lato}
  \renewcommand{\familydefault}{\sfdefault}
\else
  % If using pdflatex:
  \usepackage[rm]{roboto}
  \usepackage[defaultsans]{lato}
  % \usepackage{sourcesanspro}
  \renewcommand{\familydefault}{\sfdefault}
\fi

% Change the colours if you want to
\definecolor{SlateGrey}{HTML}{2E2E2E}
\definecolor{LightGrey}{HTML}{555555}
\definecolor{DarkPastelBlue}{HTML}{083f5d}
\definecolor{PastelBlue}{HTML}{0b4f70}
\colorlet{name}{black}
\colorlet{tagline}{PastelBlue}
\colorlet{heading}{DarkPastelBlue}
\colorlet{headingrule}{DarkPastelBlue}
\colorlet{subheading}{PastelBlue}
\colorlet{accent}{PastelBlue}
\colorlet{emphasis}{SlateGrey}
\colorlet{body}{LightGrey}

% Change some fonts, if necessary
\renewcommand{\namefont}{\huge\rmfamily\bfseries}
\renewcommand{\personalinfofont}{\footnotesize}
\renewcommand{\cvsectionfont}{\Large\rmfamily\bfseries}
\renewcommand{\cvsubsectionfont}{\large\bfseries}


% Change the bullets for itemize and rating marker
% for \cvskill if you want to
\renewcommand{\itemmarker}{{\small\textbullet}}
\renewcommand{\ratingmarker}{\faCircle}

% Contains your publications
% \addbibresource{publications.bib}

\begin{document}
\name{Savindi Wijenayaka}
\tagline{Machine Learning Engineer \& Researcher}
%% You can add multiple photos on the left or right
% \photoR{3cm}{profile-photo}

\personalinfo{%
  {
  \printinfo{\faPhone}{+64 22 453 8372}
  \printinfo{\faAt}{savindi.narmada@gmail.com}[mailto:savindi.narmada@gmail.com]
  \printinfo{\faMapMarker*}{Auckland, New Zealand}\\
  \printinfo{\faGlobe}{savindi.com}[https://savindi.com]
  \printinfo{\faLinkedin}{linkedin.com/in/savindi}[https://linkedin.com/in/savindi]
  \printinfo{\faMedium}{savindi-wijenayaka.medium.com}[https://savindi-wijenayaka.medium.com]
  }
}

\makecvheader{}

\medskip

\cvsection{Summary}

Machine Learning Engineer and Researcher with 2+ years of industry experience developing and deploying scalable deep learning systems in cloud-native environments. PhD (under examination) from the University of Auckland, with research spanning medical imaging, deep learning, applied mathematics, and computational analysis. Brings 6+ years of experience in Python, along with extensive work across modern ML frameworks, containerisation, orchestration, and CI/CD pipelines. Focused on translating research into robust, real-world AI solutions.

\medskip

\cvsection{Experience}

\cvevent{Machine Learning Engineer}{\href{http://www.wso2.com}{WSO2} $\cdot$ Full-time}{Sept 2020 - Nov 2021}{Colombo, Sri Lanka} 
WSO2 is one of the world's leading open-source integration vendors. Choreo is its latest product, providing an AI-enhanced integrated platform as a service.
\medskip
\begin{itemize} 
    \item Researched, developed, and deployed phase 1 of Choreo’s AI Test Assistant service end-to-end, including dataset creation via XML/OpenAPI scraping, iterative hypothesis testing, and model development using Python, Keras, and GPT-based experimentation for natural language understanding. Built the infrastructure with Flask, Docker, Kubernetes, and Azure, incorporating CI/CD, unit tests, and OpenAPI compliance.
    \item Co-architected and implemented Choreo’s AI Anomaly Detection system using Azure Stream Analytics, Event Hubs, SQL, and Function Apps; led the design of the alerting pipeline, including suppression policies and metric correlation.
    \item Automated Choreo’s performance testing by building a Python library and Azure DevOps pipeline for system metrics collection and analysis, improving observability, and supporting data collection for ML models.
    \item Diagnosed and resolved a critical memory leak in the Ballerina Language Server using JMeter and Eclipse Memory Analyser (MAT), which helped in the optimisation of resources in Choreo.
\end{itemize}

\divider{}

\cvevent{Software Research Engineer}{\href{https://www.pearson.com/}{Pearson} $\cdot$ Internship}{Sept 2018 - Sept 2019}{Colombo, Sri Lanka}   % chktex 8
Pearson is a leading Education provider, offering curriculum materials, multimedia learning tools, and testing programs to help educate people worldwide.
\medskip
\begin{itemize}
    \item Researched and implemented deep learning modules for emotion detection and speech analysis in the AI-based Public Speaking Evaluator Service, using Keras, Kaldi, and OpenCV. Integrated models into RESTful services (Flask, later Django) and automated deployment via Ansible for scalable real-time evaluation.

    \item Examined Question Answering and Machine Comprehension to build a Q\&A chatbot service for Pearson’s educational content, using AllenNLP and BiDAF, later extending it with fine-tuned BERT and GPT-2 models, exposing the service via a Django REST API and automated deployment using Ansible.

    \item Investigated and developed a flashcard classification service using ULMFiT, LSTM, and GRU to automatically categorise flashcards created by students or the system, deployed via Django REST framework.

    \item Evaluated and tested NoSQL and relational database migration strategies (MongoDB, MSSQL, MySQL) and conducted performance testing on ScaleOut State Servers within AWS EC2 environments.

\end{itemize}

\medskip

\cvsection{Education}

\cvevent{Ph.D. in Bioengineering \small{(under examination)}}{University of Auckland}{Dec 2021 - May 2025}{Auckland, New Zealand}   
A deep learning centred computational framework was developed for automated 3D gastric microstructure analysis, integrating biomedical imaging, model-based tissue segmentation, and downstream computational quantification to enable scalable, reproducible anatomical assessment and modelling.
\begin{itemize}
    \item Engineered a semantic segmentation model to distinguish gastric tissue layers, integrating multiscale channel and spatial attention concepts, implementing numerous ablation studies, and saving over 40 hours per dataset.
    \item Developed a robust three-dimensional gastric tissue quantification framework using advanced mathematical techniques in Python; delivered precise measurements from 20+ tissue samples, establishing the first benchmark for future research.
    \item Developed a comprehensive computational model compiling crucial geometric information alongside quantification details sourced from 8 distinct experiments, enabling future in-silico experiments.
    \item Conducted biological experiments to collect and prepare rodent stomachs for micro CT imaging, resulting in a streamlined process that enhanced sample quality and consistency across 15 experimental trials.
\end{itemize}

\divider{}

\cvevent{B.Sc. (Hons.) in Software Engineering}{University of Kelaniya}{Feb 2016 - Mar 2020}{Kelaniya, Sri Lanka}   % chktex 8
\begin{itemize}
    \item Specialised in Data Science and Net-centric application development
    \item Attained a GPA of 3.96 out of 4.00, obtaining a First Class.
\end{itemize}

\medskip

% % \pagebreak
% \cvsection{Certifications}

% \begin{itemize}
%     \item \cvevent{AI for Medical Diagnosis}{DeepLearning.AI}{May 2021}{}
%     \item \cvevent{Deep Learning Specialisation}{DeepLearning.AI}{Dec 2020}{}
%     \item \cvevent{TensorFlow Developer Specialisation}{DeepLearning.AI}{July 2020}{}
% \end{itemize}

\medskip

\cvsection{Technical Skills}
\begin{itemize}
    \item \textbf{Knowledge Areas:} Deep Learning (Computer Vision \& Natural Language Processing)
    \item \textbf{Languages \& Frameworks:} Python (Pytorch, Keras, Flask, Django), Java (Spring Boot), Ballerina, Bash
    \item \textbf{Backend \& APIs:} RESTful APIs, gRPC, Event-Driven Integration
    \item \textbf{Databases \& Data Handling:} Kusto Query Language (KQL), MongoDB, MSSQL, MySQL, ADX, DVC
    \item \textbf{Cloud Platforms \& DevOps:} Azure, Kubernetes, Docker, CI/CD, AWS, Ansible, Linux
    \item \textbf{Tools \& Methodologies:} Git, Agile, Performance Monitoring (Seaborn, Plotly, JMeter), Debugging (Eclipse MAT), Analysis (Numpy, Pandas), Testing (Unittest)
\end{itemize}

\medskip

\cvsection{Knowledge Sharing \& Technical Outreach}
\begin{itemize}
    \item Member of the teaching team for Code In Place 2021, an online Python course offered by Stanford University, contributing to global tech education initiatives.
    \item Served as a guest speaker for multiple technical webinars (organized by IEEE and DeepLearning.AI), effectively communicating complex deep learning topics to broader audiences.
    \item  Authored technical articles on Medium covering conceptual topics (CNNs, Kubernetes internals, JVM), practical applications of cloud-native microservices (Kubernetes, Docker, Azure), and automated CI/CD pipelines (GitHub Actions, Azure ARM templates), demonstrating a passion for knowledge sharing.
\end{itemize}

\medskip

\cvsection{Achievements}
\begin{itemize}
    % Technical / Academic
    \item \textbf{1st Place (2022)} and \textbf{2nd Place (2024)} in the international SPARC FAIR Codeathon, representing the University of Auckland, organised by the SPARC Data and Resource Centre and the NIH.
    \item \textbf{4th Place} in DataStorm 2020 Datathon, organised by Octave (JKH) and University of Moratuwa.
    \item \textbf{1st Runner-Up} in National Youth Software Competition 2017, organised by UNDP Sri Lanka.
    \item \textbf{Dean’s List Honouree}, recognised in all four academic years of the B.Sc. programme.

    % Leadership & Management
    \item \textbf{Vice President} of Marketing \& Communications at AIESEC in the University of Kelaniya in 2018, contributing to local chapter growth.

    % % Sports & Extracurricular
    % \item \textbf{National and University Colours} recipient, \textbf{Black Belt} Karate player and \textbf{Women's Captain} in University Karate Team; won multiple national and inter-university medals (2016–2018).

    % \item \textbf{Gold Medalist} in Chess at Four Nations Championship 2019, organised by Pearson.
    % \item Represented school at the national level in Karate, Carrom, Kabaddi, and Army Cadet competitions, while also serving as Junior Prefect, Senior Prefect, and Karate Captain.

\end{itemize}



\medskip



% \cvsection{Languages}

% \cvskill{English}{4}
% \cvskill{Sinhala}{5}

% \medskip
% \cvsection{References}
% \textit{Available upon request.}

% \vfill
% \begin{flushright}
% \flushright
% \footnotesize{\emph{Last updated: ~\today }}
% \end{flushright}

\end{document}
